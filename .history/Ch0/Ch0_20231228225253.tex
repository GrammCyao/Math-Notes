
\chapter{Foundation}

\section{Na\"ive set theory}

We begin with na\"ive set theory. A \emph{set} $A$ is an unordered collection of objects. If $x$ is an object and if $x$ lies in $A$, we say that $x$ is an element of $A$, and we write
    \begin{align*}
        x \in A,
    \end{align*}
otherwise we write
    \begin{align*}
        x \notin A.
    \end{align*}

If $A, B$ are sets and if every element of $A$ is an element of $B$, we say that $A$ is a \emph{subset} of $B$, and we write
    \begin{align*}
        A \subset B.
    \end{align*}


\begin{axiom}[Axiom of extension]
    Two sets $A$ and $B$ are equal, denoted $A = B$, if and only if they have same elements.
\end{axiom}

\begin{axiom}[Axiom of specification]
    Let $A$ be a set, and for each $x \in A$, let $P(x)$ be a property pertaining to $x$. Then there exists a set 
    \begin{align*}
        \{x \in A : P(x) \text{ is true}\},
    \end{align*}
    whose elements are precisely those elements $x$ in $A$ for which $P(x)$ is true. In other words, for any object $y$,
        \begin{align*}
            y \in \{x \in A : P(x) \text{ is true}\}
            \Longleftrightarrow 
            (y \in A \text{ and } P(y) \text{ is true}).
        \end{align*}
\end{axiom}


The axiom of specification assures the existence of empty set which without any elements at all. If $A$ is a set. The \emph{empty set} $\emptyset$ is defined to be the set
        \begin{align*}
            \emptyset := \{x \in A : x \neq x\}.
        \end{align*}
The axiom of extension implies that there can be only one set with no elements. Thus the empty set is well-defined.


\begin{axiom}[Axiom of pairing]
    Let $a$ and $b$ be objects. Then there exists a set $\{a, b\}$ whose only elements are $a$ and $b$.
\end{axiom}

If $A$ is a set such that $a \in A$ and $b \in A$ for arbitrary objects $a, b$, then we can apply the axiom of specification to $A$ with the property ``$x = a$ or $x = b$''. We obtain a set $\{x \in A : x = a \text{ or } x = b\}$ which contains just $a$ and $b$, the axiom of extension assures that such a set is unique, such a set is called the \emph{pair}. Similarly, $\{a, a\}$ is an unordered pair, denoted $\{a\}$, and is called the \emph{singleton}.

\begin{axiom}
    Let $A$ be a set. Then there exists a set 
\end{axiom}
