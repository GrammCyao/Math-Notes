
\chapter{Foundation}

\section{Sets}

\begin{comment}
We begin with na\"ive set theory. A \emph{set} $A$ is an unordered collection of objects. If $x$ is an object and if $x$ lies in $A$, we say that $x$ is an element of $A$, and we write
    \begin{align*}
        x \in A,
    \end{align*}
otherwise, we write
    \begin{align*}
        x \notin A.
    \end{align*}

If $A, B$ are sets and if every element of $A$ is an element of $B$, we say that $A$ is a \emph{subset} of $B$, and we write
    \begin{align*}
        A \subset B.
    \end{align*}
\end{comment}

We present one standard way to define the sets, in terms of the Zermelo-Fraenkel axioms, which were named after Ernst Zermelo (1871 - 1952) and Abraham Fraenkel (1891 - 1965).

\begin{axiom}[Axiom of extension]
    Two sets $A$ and $B$ are equal, denoted $A = B$, if and only if every element of $A$ is an element of $B$ and vice versa.
\end{axiom}

The axiom of extension including two fundamental concepts that of set and of element. The axiom of extension assumes that every mathematical object is a set, an element is also an object, it is meaningful to ask whether a set is an element of a set.

If $x$ is an object, we say that $x$ is an element of $A$, and we write
    \begin{align*}
        x \in A,
    \end{align*}
otherwise, we write
    \begin{align*}
        x \notin A.
    \end{align*}

Notice that we always use the \emph{first-order logic} as our meta-language, thus the axiom of extension can be rewritten as
    \begin{align*}
        \forall A\forall B(\forall x(x \in A \Longleftrightarrow x \in B) \Longleftrightarrow A = B).
    \end{align*}
But we will not express the statement in this form.

Since we obtain some fundamental concepts, we can define the \emph{subsets}.

\begin{definition}[Subsets]
    Let $A, B$ be sets. We say that $A$ is a \emph{subset} of $B$, denoted $A \subset B$, if and only if every element of $A$ is also an element of $B$.
\end{definition}

Some basic facts about subsets is given by following.

\begin{proposition}
    Let $A, B$ and $C$ be sets.
    \begin{enumerate}
        \item (Reflexivity) $A \subset A$.
        \item (Transitivity) If $A \subset B$ and $B \subset C$, then $A \subset C$.
        \item (Anti-symmetry) If $A \subset B$ and $B \subset A$, then $A = B$.
    \end{enumerate}
\end{proposition}

\begin{proof}
    Proof omitted.
\end{proof}

\begin{axiom}[Axiom scheme of specification]
    Let $A$ be a set, and for each $x \in A$, let $P(x)$ be a property pertaining to $x$. Then there exists a set 
    \begin{align*}
        \{x \in A : P(x) \text{ is true}\},
    \end{align*}
    whose elements are precisely those elements $x$ in $A$ for which $P(x)$ is true. In other words, for any object $y$,
        \begin{align*}
            y \in \{x \in A : P(x) \text{ is true}\}
            \Longleftrightarrow 
            (y \in A \text{ and } P(y) \text{ is true}).
        \end{align*}
\end{axiom}

This axiom is also known as the \emph{axiom scheme of separation}.

The axiom of specification assures the existence of empty set which without any elements at all.

\begin{definition}[Empty set]
    If $A$ is a set, the \emph{empty set} $\emptyset$ is defined to be the set
    \begin{align*}
        \emptyset := \{x \in A : x \neq x\}.
    \end{align*}
\end{definition}

Since every object $x$ must obeys $x = x$, thus for arbitrary object $x$ we have $x \notin \emptyset$, the empty set contains no elements. The axiom of extension implies that there can be only one set with no elements.

\begin{lemma}
    If $\emptyset_1$ and $\emptyset_2$ are empty set, then $\emptyset_1 = \emptyset_2$.
\end{lemma}

\begin{proof}
    Suppose for sake of contradiction that $\emptyset_1 \neq \emptyset_2$. Then by the axiom of extension, we either have there is an element $x$ such that $x \in \emptyset_1$ but $x \notin \emptyset_2$ or $x \in \emptyset_2$ but $x \notin \emptyset_1$. But both cases are impossible by the definition, a contradiction.
\end{proof}

If a set is not equal to the empty set, we call it \emph{non-empty}. The next Lemma asserts that given any non-empty set $A$, we are allowed to ``choose'' an element $x$ of $A$ which demonstrates this non-emptyness.

\begin{lemma}[Single choice]
    Let $A$ be a non-empty set. Then there exists an object $x$ such that $x \in A$.
\end{lemma}

\begin{proof}
    We prove by contradiction. Suppose there does not exist any object $x$ such that $x \in A$. Then for all objects $x$, $x \notin A$. Since we have $x \notin \emptyset$. Thus $x \in A$ equivalent to $x \in \emptyset$, and so $A = \emptyset$ by the axiom of extension.
\end{proof}

Another basic fact about the empty set is that it is a subset of every set.
\begin{lemma}
    If $A$ is a set, then $\emptyset \subset A$. In particular, we have $\emptyset \subset \emptyset$.
\end{lemma}

\begin{proof}
    Suppose for sake of contradiction that $\emptyset \subset A$ is false. This means that there is an element of $\emptyset$ which doesn't lies in $A$. But it is impossible for that $\emptyset$ contains no elements, a contradiction. In particular, since the emptyset is a set, hence $\emptyset \subset \emptyset$.
\end{proof}


\begin{axiom}[Axiom of pairing]
    Let $a$ and $b$ be objects. Then there exists a set $\{a, b\}$ whose only elements are $a$ and $b$.
\end{axiom}

If $A$ is a set such that $a \in A$ and $b \in A$ for arbitrary objects $a, b$, then we can apply the axiom of specification to $A$ with the property ``$x = a$ or $x = b$''. We obtain a set $\{x \in A : x = a \text{ or } x = b\}$ which contains just $a$ and $b$, the axiom of extension assures that such a set is unique, such a set is called the \emph{pair}. Similarly, $\{a, a\}$ is an unordered pair, denoted $\{a\}$, and is called the \emph{singleton}.

\begin{axiom}[Axiom of union]
    Let $A$ be a set, all of whose elements are sets. Then there exists a set $\bigcup A$ whose elements are precisely those objects which are elements of the elements of $A$. In other words, for any object $x$,
        \begin{align*}
            x \in \bigcup A
            \Longleftrightarrow
            (x \in S \text{ for some } S \in A).
        \end{align*}
\end{axiom}

\begin{comment}
If $A$ is a set which elements are sets, we usually call it a \emph{collection} instead of a set. By the axiom of union, for every collection $A$ there exists a set $U$ such that if $x \in S$ for some $S \in A$, then $x \in U$. We can apply the axiom of specification to form the set
    \begin{align*}
        \{x \in U : x \in S \text{ for some } S \in A\},
    \end{align*}
note that the axiom of extension guarantees its uniqueness.
\end{comment}

Now we can define some important operations on sets, namely unions, intersections and difference sets.
\begin{comment}
The \emph{unions} $S_1 \cup S_2$ of two sets is defined to be
    \begin{align*}
        S_1 \cup S_2 := \bigcup\{S_1, S_2\}.
    \end{align*}
This equivalent to say that for each $x \in S_1 \cup S_2$ we have $x \in S_1$ or $x \in S_2$. Observe that the existence and uniqueness of $S_1 \cup S_2$ are guaranteed by the axioms of pairing, union, and extension.
\end{comment}

\begin{definition}[Unions]
    The \emph{union} $S_1 \cup S_2$ of two sets is defined to be the set
    \begin{align*}
        S_1 \cup S_2 := \bigcup\{S_1, S_2\}.
    \end{align*}
\end{definition}
This equivalent to say that for each $x \in S_1 \cup S_2$ we have $x \in S_1$ or $x \in S_2$. Observe that the existence and uniqueness of $S_1 \cup S_2$ are guaranteed by the axioms of pairing, union, and extension. Here are some basic facts about unions.

\begin{proposition}[Basic properties of unions]
    Let $A, B$ be sets.
    \begin{enumerate}
        \item (Minimality) $A \cup \emptyset = A$.
        \item (Commutativity) $A \cup B = B \cup A$.
        \item (Associativity) $A \cup (B \cup C) = (A \cup B) \cup C$.
        \item (Idempotence) $A \cup A = A$.
        \item $A \subset B$ if and only if $A \cup B = B$.
    \end{enumerate}
\end{proposition}

\begin{proof}
Proof omitted.
%(i) Suppose that $x \in A \cup \emptyset$. Then we either have $x \in A$ or $x \in \emptyset$. Since $x \notin \emptyset$, we only have $x \in A$. Thus $A \cup \emptyset \subset A$. Conversely, suppose $x \in A$. Then $x \in A \cup \emptyset$ by definition.
\end{proof}

From the axiom scheme of specification, we can define the intersections.
\begin{definition}[Intersections]
    The \emph{intersection} $S_1 \cap S_2$ of two sets is defined to be the set
    \begin{align*}
        S_1 \cap S_2 := \{x \in S_1 : x \in S_2\}.
    \end{align*}
\end{definition}

Here are some basic facts about unions.

\begin{proposition}[Basic properties of intersections]
    Let $A, B$ be sets.
    \begin{enumerate}
        \item (Minimality) $A \cap \emptyset = \emptyset$.
        \item (Commutativity) $A \cap B = B \cap A$.
        \item (Associativity) $A \cap (B \cap C) = (A \cap B) \cap C$.
        \item (Idempotence) $A \cap A = A$.
        \item $A \subset B$ if and only if $A \cap B = A$.
    \end{enumerate}
\end{proposition}

\begin{proof}
    Proof omitted.
\end{proof}

\begin{proposition}[Distributive laws]
    Let $A, B$ and $C$ be sets. Then
        \begin{align*}
            A \cap (B \cup C) &= (A \cap B) \cup (A \cap C),\\
            A \cup (B \cap C) &= (A \cup B) \cap (A \cup C).
        \end{align*}
\end{proposition}

\begin{proof}
    We only prove the first one, the second one is similar. If $x \in A \cap (B \cup C)$, then $x \in A$ and $x \in (B \cup C)$. Since $x \in (B \cup C)$ implies $x \in B$ or $x \in C$. If $x \in B$, then we have $x \in A$ and $x \in B$, and conclude that $x \in A \cap B$. If $x \in C$, then we have $x \in A$ and $x \in C$, and conclude that $x \in A \cap C$. Since we either have $x \in B$ or $x \in C$, thus we either have $x \in A \cap B$ or $x \in A \cap C$. Together our conclusions, we have $x \in (A \cap B) \cup (A \cap C)$. To prove the reverse, we see that either $x \in A \cap B$ or $x \in A \cap C$, this implies we either have both $x \in A$ and $x \in B$ or both $x \in A$ and $x \in C$. Since $x \in A$ is always true, we either have $x \in B$ or $x \in C$. Thus, we conclude that $x \in A$, and $x \in B$ or $x \in C$, i.e., $x \in A \cap (B \cup C)$.
\end{proof}

Similarly, the difference sets can be defined with the axiom scheme of specification.

\begin{definition}[Difference sets]
    The \emph{difference set} $A \setminus B$ (or $A - B$) is defined to be the set $A$ with any elements of $B$ removed:
    \begin{align*}
        A \setminus B := \{x \in A : x \notin B\}.
    \end{align*}
\end{definition}

The basic facts about difference sets can be described as follows.

\begin{proposition}[Basic properties of difference sets]
    Let $A, B, X$ be sets, and let $A, B$ be subsets of $X$.
    \begin{enumerate}
        \item $X \setminus (X \setminus A) = A$.
        \item $X \setminus \emptyset = X$ and $X \setminus X = \emptyset$.
        \item $A \cap (X \setminus A) = \emptyset$ and $A \cup (X \setminus A) = X$.
        \item $A \subset B$ if and only if $X \setminus B \subset X \setminus A$.
    \end{enumerate}
\end{proposition}

\begin{proof}
    Proof omitted.
\end{proof}

\begin{proposition}[De Morgan laws]
    Let $A, B, X$ be sets, and let $A, B$ be subsets of $X$. Then
    \begin{align*}
        X \setminus (A \cup B) &= (X \setminus A) \cap (X \setminus B),\\
        X \setminus (A \cap B) &= (X \setminus A) \cup (X \setminus B).
    \end{align*}
\end{proposition}

\begin{proof}
    Proof omitted.
\end{proof}

There is a handy operation of sets which is usually used in mathematical proofs, namely symmetric difference.

\begin{definition}[Symmetric difference]
    The \emph{symmetric difference} $A \triangle B$ of two sets is defined to be the set
    \begin{align*}
        A \triangle B := (A \setminus B) \cup (B \setminus A).
    \end{align*}
\end{definition}

There is some properties of symmetric differences.

\begin{proposition}[Basic properties of symmetric differences]
    Let $A, B, C$ and $X$ be sets, and let $A \subset X$.
    \begin{enumerate}
        \item (Commutativity) $A \triangle B = B \triangle A$.
        \item (Associativity) $A \triangle (B \triangle C) = (A \triangle B) \triangle C$.
        \item $A \cap (B \triangle C) = (A \cap B) \triangle (A \cap C)$.
        \item $A \triangle \emptyset = A$ and $A \triangle X = X \setminus A$.
        \item $A \triangle A = \emptyset$ and $A \triangle (X \setminus A) = X$.
        \item $A \triangle B = (A \cup B) \setminus (A \cap B)$.
    \end{enumerate}
\end{proposition}

\begin{proof}
    Proof omitted.
\end{proof}

\begin{axiom}[Axiom of powers]
    Let $X$ be a set. Then the set
    \begin{align*}
        \{Y : Y \subset X\}
    \end{align*}
    is a set, such a set is called the \emph{power set} of $X$.
\end{axiom}

\section{Cartesian product}

Another fundamental operation on sets, the \emph{Cartesian product}. We first define the \emph{ordered pair}.

\begin{definition}[Ordered pair]
    If $a$ and $b$ are any objects, we define the \emph{ordered pair} $(a, b)$ to be the set
    \begin{align*}
        (a, b) := \{\{a\}, \{a, b\}\}.
    \end{align*}
\end{definition}

We prove that two pairs are equal if both their components match.

\begin{lemma}
    If $(a, b)$ and $(x, y)$ are ordered pairs and if $(a, b) = (x, y)$, then $a = x$ and $b = y$.
\end{lemma}

\begin{proof}
    We first show that $(a, b)$ is singleton if and only if $a = b$. If $(a, b)$ is singleton, then $\{a\} = \{a, b\}$, so that $b \in \{a\}$, and hence $b = a$. Conversely, if $$
\end{proof}













