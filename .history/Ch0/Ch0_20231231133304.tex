
\chapter{Foundation}

\section{Sets}

\begin{comment}
We begin with na\"ive set theory. A \emph{set} $A$ is an unordered collection of objects. If $x$ is an object and if $x$ lies in $A$, we say that $x$ is an element of $A$, and we write
    \begin{align*}
        x \in A,
    \end{align*}
otherwise, we write
    \begin{align*}
        x \notin A.
    \end{align*}

If $A, B$ are sets and if every element of $A$ is an element of $B$, we say that $A$ is a \emph{subset} of $B$, and we write
    \begin{align*}
        A \subset B.
    \end{align*}
\end{comment}

We present one standard way to define the sets, in terms of the axiomatic set theory.

\begin{axiom}[Axiom of extension]\label{axim:extension}
    Two sets $A$ and $B$ are equal, denoted $A = B$, if and only if every element of $A$ is an element of $B$ and vice versa.
\end{axiom}

The axiom of extension including two fundamental concepts that of set and of element. The axiom of extension assumes that every mathematical object is a set, an element is also an object, it is meaningful to ask whether a set is an element of a set.

If $x$ is an object, we say that $x$ is an element of $A$, and we write
    \begin{align*}
        x \in A,
    \end{align*}
otherwise, we write
    \begin{align*}
        x \notin A.
    \end{align*}

Notice that we always use the \emph{first-order logic} as our meta-language, thus the axiom of extension can be rewritten as
    \begin{align*}
        \forall A\forall B(\forall x(x \in A \Longleftrightarrow x \in B) \Longleftrightarrow A = B).
    \end{align*}
But we shall not express the statement in this form.

Since we obtain some fundamental concepts, we can define the \emph{subsets}.

\begin{definition}[Subsets]
    Let $A, B$ be sets. We say that $A$ is a \emph{subset} of $B$, denoted $A \subset B$, if and only if every element of $A$ is also an element of $B$.
\end{definition}

Some basic facts about subsets is given by following.

\begin{proposition}
    Let $A, B$ and $C$ be sets.
    \begin{enumerate}
        \item (Reflexivity) $A \subset A$.
        \item (Transitivity) If $A \subset B$ and $B \subset C$, then $A \subset C$.
        \item (Anti-symmetry) If $A \subset B$ and $B \subset A$, then $A = B$.
    \end{enumerate}
\end{proposition}

\begin{proof}
    Proof omitted.
\end{proof}

\begin{axiom}[Axiom scheme of specification]
    Let $A$ be a set, and for each $x \in A$, let $P(x)$ be a property pertaining to $x$. Then there exists a set 
    \begin{align*}
        \{x \in A : P(x) \text{ is true}\},
    \end{align*}
    whose elements are precisely those elements $x$ in $A$ for which $P(x)$ is true. In other words, for any object $y$,
        \begin{align*}
            y \in \{x \in A : P(x) \text{ is true}\}
            \Longleftrightarrow 
            (y \in A \text{ and } P(y) \text{ is true}).
        \end{align*}
\end{axiom}

This axiom is also known as the \emph{axiom scheme of separation}.

The axiom of specification assures the existence of empty set which without any elements at all.

\begin{definition}[Empty set]
    If $A$ is a set, the \emph{empty set} $\emptyset$ is defined to be the set
    \begin{align*}
        \emptyset := \{x \in A : x \neq x\}.
    \end{align*}
\end{definition}

Since every object $x$ must obeys $x = x$, thus for arbitrary object $x$ we have $x \notin \emptyset$, the empty set contains no elements. The axiom of extension implies that there can be only one set with no elements.

\begin{lemma}
    If $\emptyset_1$ and $\emptyset_2$ are empty set, then $\emptyset_1 = \emptyset_2$.
\end{lemma}

\begin{proof}
    Suppose for sake of contradiction that $\emptyset_1 \neq \emptyset_2$. Then by the axiom of extension, we either have there is an element $x$ such that $x \in \emptyset_1$ but $x \notin \emptyset_2$ or $x \in \emptyset_2$ but $x \notin \emptyset_1$. But both cases are impossible by the definition, a contradiction.
\end{proof}

If a set is not equal to the empty set, we call it \emph{non-empty}. The next Lemma asserts that given any non-empty set $A$, we are allowed to ``choose'' an element $x$ of $A$ which demonstrates this non-emptyness.

\begin{lemma}[Single choice]
    Let $A$ be a non-empty set. Then there exists an object $x$ such that $x \in A$.
\end{lemma}

\begin{proof}
    We prove by contradiction. Suppose there does not exist any object $x$ such that $x \in A$. Then for all objects $x$, $x \notin A$. Since we have $x \notin \emptyset$. Thus $x \in A$ equivalent to $x \in \emptyset$, and so $A = \emptyset$ by the axiom of extension.
\end{proof}

Another basic fact about the empty set is that it is a subset of every set.
\begin{lemma}
    If $A$ is a set, then $\emptyset \subset A$. In particular, we have $\emptyset \subset \emptyset$.
\end{lemma}

\begin{proof}
    Suppose for sake of contradiction that $\emptyset \subset A$ is false. This means that there is an element of $\emptyset$ which doesn't lies in $A$. But it is impossible for that $\emptyset$ contains no elements, a contradiction. In particular, since the emptyset is a set, hence $\emptyset \subset \emptyset$.
\end{proof}


\begin{axiom}[Axiom of pairing]
    Let $a$ and $b$ be objects. Then there exists a set $\{a, b\}$ whose only elements are $a$ and $b$.
\end{axiom}

If $A$ is a set such that $a \in A$ and $b \in A$ for arbitrary objects $a, b$, then we can apply the axiom of specification to $A$ with the property ``$x = a$ or $x = b$''. We obtain a set $\{x \in A : x = a \text{ or } x = b\}$ which contains just $a$ and $b$, the axiom of extension assures that such a set is unique, such a set is called the \emph{pair}. Similarly, $\{a, a\}$ is an unordered pair, denoted $\{a\}$, and is called the \emph{singleton}.

\begin{axiom}[Axiom of union]
    Let $A$ be a set, all of whose elements are sets. Then there exists a set $\bigcup A$ whose elements are precisely those objects which are elements of the elements of $A$. In other words, for any object $x$,
        \begin{align*}
            x \in \bigcup A
            \Longleftrightarrow
            (x \in S \text{ for some } S \in A).
        \end{align*}
\end{axiom}

\begin{comment}
If $A$ is a set which elements are sets, we usually call it a \emph{collection} instead of a set. By the axiom of union, for every collection $A$ there exists a set $U$ such that if $x \in S$ for some $S \in A$, then $x \in U$. We can apply the axiom of specification to form the set
    \begin{align*}
        \{x \in U : x \in S \text{ for some } S \in A\},
    \end{align*}
note that the axiom of extension guarantees its uniqueness.
\end{comment}

Now we can define some important operations on sets, namely unions, intersections and difference sets.
\begin{comment}
The \emph{unions} $S_1 \cup S_2$ of two sets is defined to be
    \begin{align*}
        S_1 \cup S_2 := \bigcup\{S_1, S_2\}.
    \end{align*}
This equivalent to say that for each $x \in S_1 \cup S_2$ we have $x \in S_1$ or $x \in S_2$. Observe that the existence and uniqueness of $S_1 \cup S_2$ are guaranteed by the axioms of pairing, union, and extension.
\end{comment}

\begin{definition}[Unions]
    The \emph{union} $S_1 \cup S_2$ of two sets is defined to be the set
    \begin{align*}
        S_1 \cup S_2 := \bigcup\{S_1, S_2\}.
    \end{align*}
\end{definition}
This equivalent to say that for each $x \in S_1 \cup S_2$ we have $x \in S_1$ or $x \in S_2$. Observe that the existence and uniqueness of $S_1 \cup S_2$ are guaranteed by the axioms of pairing, union, and extension. Here are some basic facts about unions.

\begin{proposition}[Basic properties of unions]
    Let $A, B$ be sets.
    \begin{enumerate}
        \item (Minimality) $A \cup \emptyset = A$.
        \item (Commutativity) $A \cup B = B \cup A$.
        \item (Associativity) $A \cup (B \cup C) = (A \cup B) \cup C$.
        \item (Idempotence) $A \cup A = A$.
        \item $A \subset B$ if and only if $A \cup B = B$.
    \end{enumerate}
\end{proposition}

\begin{proof}
Proof omitted.
%(i) Suppose that $x \in A \cup \emptyset$. Then we either have $x \in A$ or $x \in \emptyset$. Since $x \notin \emptyset$, we only have $x \in A$. Thus $A \cup \emptyset \subset A$. Conversely, suppose $x \in A$. Then $x \in A \cup \emptyset$ by definition.
\end{proof}

From the axiom scheme of specification, we can define the intersections.
\begin{definition}[Intersections]
    The \emph{intersection} $S_1 \cap S_2$ of two sets is defined to be the set
    \begin{align*}
        S_1 \cap S_2 := \{x \in S_1 : x \in S_2\}.
    \end{align*}
\end{definition}

Here are some basic facts about unions.

\begin{proposition}[Basic properties of intersections]
    Let $A, B$ be sets.
    \begin{enumerate}
        \item (Minimality) $A \cap \emptyset = \emptyset$.
        \item (Commutativity) $A \cap B = B \cap A$.
        \item (Associativity) $A \cap (B \cap C) = (A \cap B) \cap C$.
        \item (Idempotence) $A \cap A = A$.
        \item $A \subset B$ if and only if $A \cap B = A$.
    \end{enumerate}
\end{proposition}

\begin{proof}
    Proof omitted.
\end{proof}

\begin{proposition}[Distributive laws]
    Let $A, B$ and $C$ be sets. Then
        \begin{align*}
            A \cap (B \cup C) &= (A \cap B) \cup (A \cap C),\\
            A \cup (B \cap C) &= (A \cup B) \cap (A \cup C).
        \end{align*}
\end{proposition}

\begin{proof}
    We only prove the first one, the second one is similar. If $x \in A \cap (B \cup C)$, then $x \in A$ and $x \in (B \cup C)$. Since $x \in (B \cup C)$ implies $x \in B$ or $x \in C$. If $x \in B$, then we have $x \in A$ and $x \in B$, and conclude that $x \in A \cap B$. If $x \in C$, then we have $x \in A$ and $x \in C$, and conclude that $x \in A \cap C$. Since we either have $x \in B$ or $x \in C$, thus we either have $x \in A \cap B$ or $x \in A \cap C$. Together our conclusions, we have $x \in (A \cap B) \cup (A \cap C)$. To prove the reverse, we see that either $x \in A \cap B$ or $x \in A \cap C$, this implies we either have both $x \in A$ and $x \in B$ or both $x \in A$ and $x \in C$. Since $x \in A$ is always true, we either have $x \in B$ or $x \in C$. Thus, we conclude that $x \in A$, and $x \in B$ or $x \in C$, i.e., $x \in A \cap (B \cup C)$.
\end{proof}

Similarly, the difference sets can be defined with the axiom scheme of specification.

\begin{definition}[Difference sets]
    The \emph{difference set} $A \setminus B$ (or $A - B$) is defined to be the set $A$ with any elements of $B$ removed:
    \begin{align*}
        A \setminus B := \{x \in A : x \notin B\}.
    \end{align*}
\end{definition}

The basic facts about difference sets can be described as follows.

\begin{proposition}[Basic properties of difference sets]
    Let $A, B, X$ be sets, and let $A, B$ be subsets of $X$.
    \begin{enumerate}
        \item $X \setminus (X \setminus A) = A$.
        \item $X \setminus \emptyset = X$ and $X \setminus X = \emptyset$.
        \item $A \cap (X \setminus A) = \emptyset$ and $A \cup (X \setminus A) = X$.
        \item $A \subset B$ if and only if $X \setminus B \subset X \setminus A$.
    \end{enumerate}
\end{proposition}

\begin{proof}
    Proof omitted.
\end{proof}

\begin{proposition}[De Morgan laws]
    Let $A, B, X$ be sets, and let $A, B$ be subsets of $X$. Then
    \begin{align*}
        X \setminus (A \cup B) &= (X \setminus A) \cap (X \setminus B),\\
        X \setminus (A \cap B) &= (X \setminus A) \cup (X \setminus B).
    \end{align*}
\end{proposition}

\begin{proof}
    Proof omitted.
\end{proof}

There is a handy operation of sets which is usually used in mathematical proofs, namely symmetric difference.

\begin{definition}[Symmetric difference]
    The \emph{symmetric difference} $A \triangle B$ of two sets is defined to be the set
    \begin{align*}
        A \triangle B := (A \setminus B) \cup (B \setminus A).
    \end{align*}
\end{definition}

There is some properties of symmetric differences.

\begin{proposition}[Basic properties of symmetric differences]
    Let $A, B, C$ and $X$ be sets, and let $A \subset X$.
    \begin{enumerate}
        \item (Commutativity) $A \triangle B = B \triangle A$.
        \item (Associativity) $A \triangle (B \triangle C) = (A \triangle B) \triangle C$.
        \item $A \cap (B \triangle C) = (A \cap B) \triangle (A \cap C)$.
        \item $A \triangle \emptyset = A$ and $A \triangle X = X \setminus A$.
        \item $A \triangle A = \emptyset$ and $A \triangle (X \setminus A) = X$.
        \item $A \triangle B = (A \cup B) \setminus (A \cap B)$.
    \end{enumerate}
\end{proposition}

\begin{proof}
    Proof omitted.
\end{proof}

\begin{axiom}[Axiom of powers]
    Let $X$ be a set. Then the set
    \begin{align*}
        \{Y : Y \subset X\}
    \end{align*}
    is a set, such a set is called the \emph{power set} of $X$.
\end{axiom}

\begin{axiom}[Axiom scheme of replacement]
    Let $A$ be a set. For any object $x \in A$, and any object, suppose we have a statement $P(x, y)$ pertaining to $x$ and $y$, such that for each $x \in A$ there is at most one $y$ for which $P(x, y)$ is true. Then there exists a set $\{y : P(x, y) \text{ is true for some } x \in A\}$, such that for any object $z$,
    \begin{align*}
        z \in \{y : P(x, y) \text{ is true for some } x \in A\}
        \Longleftrightarrow
        P(x, z) \text{ is true for some } x \in A.
    \end{align*}
\end{axiom}

\begin{axiom}[Axiom of infinity]
    There exists a set containing $0$ and containing the successor of each of its elements.
\end{axiom}

The axiom of infinity introduces the most basic example of an infinite set, namely the set of natural numbers. we shall return to it when we set about constructing the natural numbers.

\section{Russell's paradox}

\begin{pseudoaxiom}[Axiom of comprehension]
    Suppose for every object $x$ we have a property $P(x)$ pertaining to $x$. Then there exists a set $\{x : P(x) \text{ is true}\}$ such that for every object $y$,
    \begin{align*}
        y \in \{x : P(x) \text{ is true}\}
        \Longleftrightarrow
        P(y) \text{ is true}.
    \end{align*}
\end{pseudoaxiom}

This axiom is a pseudo-axiom because it creates a logical contradiction known as \emph{Russell's paradox}, discovered by the philosopher and logician Bertrand Russell (1872 - 1970) in 1901. The paradox runs as follows.

Let $P(x)$ be the statement ``$x$ is a set, and $x \notin x$'', i.e., $P(x)$ is true only when $x$ is a set which does not contain itself. If we let $S$ be the set of all sets (this is possible from the axiom of comprehension), then since $S$ is itself a set, it is an element of $S$, and so $P(x)$ is false. Now use the axiom of comprehension to create the set $\Omega := \{x : P(x)\}$. $\Omega$ is the set of all sets which do not contain themselves. We wonder that is $\Omega \in \Omega$ true? If $\Omega$ did contain itself, then by definition this means that $P(\Omega)$ is true, i.e., $\Omega$ is a set and $\Omega \notin \Omega$. Otherwise, if $\Omega$ did not contain itself, then $P(\Omega)$ would be true, and hence $\Omega \in \Omega$. Thus in either case we have both $\Omega \in \Omega$ and $\Omega \notin \Omega$, which is absurd.

We shall simply postulate an axiom which ensures that absurdities such as Russell's paradox do not occur.

\begin{axiom}[Axiom of regularity]\label{axiom:regularity}
    If $A$ is a non-empty set, then there is at least one element $x$ of $A$ which is either not a set, or is disjoint from $A$, i.e., $x \cap A = \emptyset$.
\end{axiom}

One particular consequence of this axiom is that sets are no longer allowed to contain themselves.

\begin{lemma}
    If $A$ is a set, then $A \notin A$. Furthermore, if $A$ and $B$ are two sets, then either $A \notin B$ or $B \notin A$.
\end{lemma}

\begin{proof}
Suppose that $A \neq \emptyset$ and $A \in A$. Then we have $A \in \{A\}$. Since $A$ is a set, by the axiom of regularity, we only have $A \cap \{A\} = \emptyset$. But since $A \in A$ and $A \in \{A\}$, this implies that $A \cap \{A\} = A$ which is not non-empty, a contradiction. While if $A = \emptyset$ which contains no elements, thus $A \notin A$.

Furthermore, suppose that $A, B$ are two sets and we both have $A \in B$ and $B \in A$. Then $A,B \in \{A, B\}$. By the axiom of regularity, we either have $A \cap \{A, B\} = \emptyset$ or $B \cap \{A,B\} = \emptyset$. Since $A \in B$, we have $A \cap \{A, B\} = B$, but $B$ is non-empty for that $A \in B$. With similar reasoning, $B \cap \{A,B\}$ is empty implies $B \cap \{A,B\}$ is non-empty, a contradiction.
\end{proof}

\begin{remark}
    The axioms of set theory that we have introduced (Axioms \ref{axim:extension}-\ref{axiom:regularity}) are known as the \emph{Zermelo-Fraenkel axioms} (ZF axioms) of set theory, after Ernest Zermelo (1871-1953) and Abraham Fraenkel (1891-1965). There is one further axiom we will eventually need, the famous \emph{axiom of choice} (see Section \ref{undefined}), giving rise to the \emph{Zermelo-Fraenkel-Choice axioms} (ZFC axioms) of set theory, bu we will not need this axiom for some time.
\end{remark}

\section{Cartesian product}

Another fundamental operation on sets, the \emph{Cartesian product}. We first define the \emph{ordered pair}.

\begin{definition}[Ordered pair]
    If $a$ and $b$ are any objects, we define the \emph{ordered pair} $(a, b)$ to be the set
    \begin{align*}
        (a, b) := \{\{a\}, \{a, b\}\}.
    \end{align*}
\end{definition}

We prove that two pairs are equal if and only if both their components match.

\begin{lemma}
    If $(a, b)$ and $(x, y)$ are ordered pairs, then $(a, b) = (x, y)$ if and only if $a = x$ and $b = y$.
\end{lemma}

\begin{proof}
    We first show that $(a, b)$ is singleton if and only if $a = b$. If $(a, b)$ is singleton, then $\{a\} = \{a, b\}$, so that $b \in \{a\}$, and hence $b = a$. Conversely, if $a = b$, then $(a, b) = \{\{a\}\}$ which is singleton.

    Now we return to the lemma. If $a = b$, then both $(a, b)$ and $(x, y)$ are singletons, so that $x = y$. Since $\{x\} \in (a, b)$ and $\{a\} \in (x, y)$, this implies that $a, b, x$ and $y$ are all equal.

    If $a \neq b$, then $(a, b)$ and $(x, y)$ contain exactly one singleton, i.e., $\{a\}$ and $\{x\}$. Thus $a = x$. Furthermore, $(a, b)$ and $(x, y)$ contain exactly one pair, i.e., $\{a, b\}$ and $\{x, y\}$, and thus $\{a, b\} = \{x, y\}$. Since $a = x$ and $b \in \{x, y\}$, we only have $b = y$.

    For the other direction, suppose that $a = x$ and $b = y$. We only need to show that case of $a \neq b$. By the hypothesis, we have $\{a\} = \{x\}$ and $\{a, b\} = \{x, y\}$. Thus $(a, b) = (x, y)$ is hold for that $\{\{a\}, \{a, b\}\} = \{\{x\}, \{x, y\}\}$ is hold. This complete the proof.
\end{proof}

Now we can introduce the \emph{Cartesian product}.

\begin{definition}[Cartesian product]
    If $X$ and $Y$ are sets, then we define the \emph{Cartesian product} $X \times Y$ to be the collection of ordered pairs, whose first component lies in $X$ and second component lies in $Y$, i.e.,
    \begin{align*}
        X \times Y := \{(x, y) : (x, y) \text{ for some }x \in X \text{ and } y \in Y\}.
    \end{align*}
\end{definition}

Here are some properties of Cartesian product.

\begin{proposition}
    Let $A, B, X, Y$ be sets.
    \begin{enumerate}
        \item $(A \cup B) \times X = (A \times X) \cup (B \times X)$.
        \item $(A \cap B) \times (X \cap Y) = (A \times X) \cap (B \times Y)$.
        \item $(A \setminus B) \times X = (A \times X) \setminus (B \times X)$.
        \item $(X \times Y) \setminus (A \times B) = ((X \setminus A) \times Y) \cup (X \times (Y \setminus B))$.
        \item $A \times B = \emptyset$ if and only if $A = \emptyset$ or $B = \emptyset$.
        \item If $A \subset X$ and $B \subset Y$, then $A \times B \subset X \times Y$.
        \item Let $A \times B \neq \emptyset$. If $A \times B \subset X \times Y$, then $A \subset X$ and $B \subset Y$.
        \item If $A = X$ and $B = Y$, then $A \times B = X \times Y$. Conversely, if $A \times B \neq \emptyset$ and $A \times B = X \times Y$, then $A = Y$ and $B = Y$.
    \end{enumerate}
\end{proposition}

\begin{proof}
    %(i) Suppose $(x, y) \in (A \cup B) \times X$, by definition, $(x, y)$ for some $x \in A \cup B$ and $y \in X$. Then we either have $x \in A$ or $x \in B$. If $x \in A$, then $(x, y)$ for some $x \in A$ and $y \in X$, so that $(x, y) \in A \times X$. If $x \in B$, we have $(x, y) \in B \times X$ with similar reasoning. Since we either have $(x, y) \in A \times X$ or $(x, y) \in B \times X$, thus $(x, y) \in (A \times X) \cup (B \times X)$.
    We only prove (v), (vi) and (vii).

    (v) For the ``if'' part, assume that $A \times B \neq \emptyset$. Then there is $(x, y) \in A \times B$, which means $x \in A$ and $y \in B$. So that $A \neq \emptyset$ and $B \neq \emptyset$, a contradiction. Conversely, assume that neither $A$ nor $B$ is empty, then there is $x \in A$ and $y \in B$. This implies $(x, y) \in A \times B$, a contradiction.

    (vi) Suppose $A \subset X$ and $B \subset Y$, then for every $x \in A$ and every $y \in B$, we have $(x, y) \in A \times B$. Since $x \in X$ and $y \in Y$, we implies that $(x, y) \in X \times Y$. Thus $A \times B \subset X \times Y$.

    (vii) Suppose that $(x, y) \in A \times B$. Suppose for sake of contradiction that there exists $a \in A$ but $a \notin X$. Then $(a, y) \in A \times B$ and $(a, y) \notin X \times Y$, but it is impossible for that $A \times B \subset X \times Y$. Thus $A \subset X$. A same proof shows that $B \subset Y$.

    Note that we requires $A \times B$ be non-empty, this equivalently requires that both $A$ and $B$ are non-empty from (v). If $A = \emptyset$, we must have $A \times B = \emptyset \subset X \times Y$ for arbitrary $B$. We can let $B \supset Y$ and $B \neq Y$, then the premise is true but the conclusion is false.
\end{proof}

\begin{definition}{Relations}
    A \emph{relation} $R$ is a subset of a Cartesian product $X \times Y$. We usually write $x R y$ instead of $(x, y) \in R$.

    We define the \emph{domain} $\dom{R}$ of $R$ and the \emph{range} $\ran{R}$
\end{definition}


\section{Functions}
\begin{comment}
\begin{definition}[Functions]
    Let $X, Y$ be sets, and let $P(x, y)$ be a property pertaining to an object $x \in X$ and an object $y \in Y$, such that for every $x \in X$, there is exactly one $y \in Y$ for which $P(x, y)$ is true. Then we define the \emph{function $f : X \to Y$ defined by $P$ on the domain $X$ and range $Y$} to be the object which, given any input $x \in X$, assigns an output $f(x) \in Y$, defined to be the unique object $f(x)$ for which $P(x, f(x))$ is true. Thus for any $x \in X$ and $y \in Y$,
    \begin{align*}
        y = f(x) \Longleftrightarrow P(x, y) \text{ is true}.
    \end{align*}
\end{definition}
\end{comment}






