
\chapter{Submersions, immersions, and embeddings}

\section{Maps of constant rank}

\begin{definition}
    Let $M$ and $N$ be smooth manifolds. Let $F : M \to N$ be a smooth map, and let $p \in M$.
    \begin{itemize}
        \item We define the \emph{rank of $F$ at $p$} to be the rank of the linear map $dF_p : T_pM \to T_{F(p)}N$. If $F$ has same rank $r$ at every point of $M$, then we say that it has \emph{constant rank}, and denote $\rank{F} = r$.
        \item The smooth map $F$ is called a \emph{smooth submersion} if $F_p$ is surjective at every point of $M$, i.e., $\rank{F} = \dim{N}$.
        \item The smooth map $F$ is called a \emph{smooth immersion} if $F_p$ is injective at every point of $M$, i.e., $\rank{F} = \dim{M}$.
    \end{itemize}
\end{definition}

\begin{theorem}[Rank theorem]
    Let $M$ and $N$ be smooth manifold of dimension $m$ and $n$, respectively. Let $F : M \to N$ be a smooth map with constant rank $r$. For every $p \in M$ there exist charts $(U, \varphi)$ containing $p$ and $(V, \psi)$ containing $F(p)$ 
\end{theorem}




